\documentclass{article}
\usepackage[utf8]{inputenc}
\usepackage{graphics}
\usepackage[paper=a4paper,left=3cm, right=3cm, top=2cm]{geometry}
\usepackage{blindtext}
\usepackage{multicol}
\usepackage{pgfplots}
\usepackage{float}
\usepackage{biblatex} %Imports biblatex package
\usepackage{amsmath}
\addbibresource{minor.bib} %Import the bibliography file


\pgfplotsset{compat = newest}


\title{%
Nuclear Medicine: An Approach Via Nanotechnology\\
   \large Department of Biomedical Engineering\\
    National Institute of Technology, Raipur}
\author{Durgesh Kumar(18111023)}
\date{January 2022}

\begin{document}

\maketitle

\begin{abstract}\textbf{The birth of nanotechnology in human society was around 2000 years ago and soon found applications in various fields and with nanotechnology nuclear medicine (radiation techniques) are an important tool in fighting against cancer and are used to treat for a variety of malignant tumors of different origins and stage; it is used in the treatment of as many as 50\% of all cancer patients and other then cancer infection/inflammation, and brain and heart diseases also treated with it.Every day in the clinic, nuclear physicians work with nuclear biologists, chemists, and physicists to practise nuclear medicine. Nanomedicine is a medical use of nanomaterials, biologic and nanoelectronic devices, biosensors, and possibly molecular nanotechnology. The fact that radiation dose can be supplied locally and that cells within the radiation nanoparticles transfer radionuclides to tumours while sparing healthy tissues from radiation induced damage account for radiation's efficacy as a cancer treatment technique.Because nanoparticles can be targeted to malignancies, nanotechnology has a promising future in radiotherapy. As a result, scientists have seized on the notion of employing nanoparticles to deliver radionuclides to tumours while avoiding radiation damage to healthy cells. Dendrimer nanocomposites have recently been discovered to be useful in radiation and imaging of tumour microvasculature. Carbon nanoparticles, also known as buckyballs, are barely a nanometer in diameter and could be used as antiradiation medications in the future to assist protect against the negative effects of cancer therapies or dirty bombs.Radiation therapy and chemotherapy frequently harm cells and tissues by releasing potentially harmful reactive oxygen species such as free radicals, oxygen ions, and peroxides. The electron clouds that surround buckyballs are thought to be capable of absorbing these free radicals. In light of a number of recent breakthroughs, the emerging role of nanotechnology in conjunction with radiation biology appears to be highly promising.}\\ 
    \\
    \textbf{\textit{Keywords-Cancer treatment; dendrimer nanocomposites;Nano Technology;Nuclear Medicine }}
\end{abstract}

\vspace{1 cm}

\begin{multicols}{2}
\section{Introduction}
nanotechnology is the manipulation of matter on an atomic and molecular scale to prepare new structures, devices and materials. In general, nanotechnology uses materials sized between 1 and 100 nanometres \cite{hong2009molecular}.The interdisciplinary nature of nanotechnology has given it a fundamental influence in various fields, especially medical science. Nanomedicine is defined as the knowledge and skill of manipulating and exploiting the unique chemical, physical, electrical, optical, and biological attributes of natural or synthesized material at the nano-sized scale for various medical purposes, such as opportune prevention, early detection, and targeted treatment of disease\cite{European,hong2009molecular,moghimi2005nanomedicine,pan2009nanomedicine,ting2010nanotargeted}.\subsection{Cancer application}The mere thought of cancer can be sufficient to elicit images of pain, loss, and
adverse side effects. Cancer may affect people at all ages, even fetuses, but risk for the more common varieties tends to increase with age (Cancer Research UK, 2007). Cancer causes about 13\% of all deaths (WHO, 2006). According to the American Cancer Society, 7.6 million people died from cancer in the world during 2007 (American Cancer Society, 2007). In the U.S. and other developed countries, the U.S. and other developed countries, cancer is presently responsible for about 25\% of all deaths (Jemal et al., 2005).

On an annual basis, cancer affects 0.5 percent of the population. Although developments in diagnostic and therapeutic technology have made it possible to detect cancer at an earlier stage and so treat instances more successfully, present diagnostic and treatment processes are far from ideal, and there is an urgent need to address the numerous flaws. Given the natural history of cancer, the time of diagnosis has a substantial impact on the prognosis, with early detection lowering the risk of morbidity and mortality.
\end{multicols}





Applications of nanotechnology in nuclear medicine can be found in areas of diagnostics,therapeutics, theranostics, and regenerative medicine, as described below (Fig. 1).



\begin{figure}[H]
    \centering
    \includegraphics[length=8cm,width=14cm]{Images/Figure(1).png}
    \label{fig:1}
\end{figure}


\begin{multicols}



\section{Radiation therapy  }
is now commonly accepted as one of the most effective forms of cancer treatment, and used for a variety of malignant tumors of different origins and stage. Radiotherapy uses a special kind of energy, ionizing energy, which is applied over a certain area that contains the tumor. The ionizing energy damages the nuclear genetic material of the cancer cell, thereby preventing it from properly multiplying. The success of radiation as a cancer treatment modality stems from the facts that radiation dose can be delivered locally and that cells within the radiation field can be killed effectively. Effectiveness of the radiation therapy is low if the tumor is located in vital regions of the body. The treatment is intrinsically toxic to the body and targets rapidly dividing cells such as those in a tumor. However, treatment is not selective and rapidly dividing cells (i.e. hair, intestinal lining, and bone-marrow) are also killed in the process. Thus, the main
limitation of radiation therapy is that they kill healthy cells almost as easily as they do
tumors.


Such limitations warrant the need for novel cancer imaging and therapeutic modalities that are increasingly safer and more specific for their cancerous targets. The current focus in development of cancer therapies (Jain, 2005) is on targeted drug delivery to provide therapeutic concentrations of anticancer agents at the site of action and to spare the normal tissues. Targeted drug delivery to tumors can increase the selectivity for killing cancer cells, decrease the peripheral/systemic toxicity and can permit a dose escalation. This will more advantageous then earlier. Now days the drug delivery system uses nanoparticale based transportation of drugs that helps to target cancerous tumors  only. It increases the efficiency of cancer treatment.

Nanotechnology can be advantageously
used to eradicate cancer cells without
harming healthy ones. Scientists hope to use nanotechnology to create therapeutic agents that target specific cancer cells and deliver the toxin and radiation in a controlled, time release manner. The ultimate goal of this research is nanoparticles that will circulate through the body, detect cancer-associated molecular changes, assist with imaging, 


Nanotechnology can be advantageously
used to eradicate cancer cells without
harming healthy ones. Scientists hope to use nanotechnology to create therapeutic agents that target specific cancer cells and deliver the toxin and radiation in a controlled, timerelease manner. The ultimate goal of this research is nanoparticles that will circulate through the body, detect cancer-associated molecular changes, assist with imaging, release a therapeutic agent, and then monitor the effectiveness of the intervention, thus enhancing the subsequent effect of radiation therapy. Nanomaterials have garnered increasing interest recently as potential therapeutic drug-delivery vehicles. Among the existing nanomaterials are the pure carbon-based particles, such as fullerenes and nanotubes, various organic dendrimers, liposomes and other polymeric compounds. These vehicles have been decorated with a wide spectrum of target-reactive ligands, such as antibodies and peptides, which interact with cell-surface tumor antigens or vascular epitopes. Once targeted, these new nanomaterials can then deliver radioisotopes or isotope generators to the cancer cells. Here, we will review some of the more common nanomaterials under investigation and their current and future applications as drug-delivery scaffolds with particular emphasis on targeted cancer radiotherapy.

\subsection{Nanowire Sensors}
Nanowires are made of carbon, silicon
and other materials that have unique
properties (NCI, 2006). When used as a
sensor, nanowires lay across a small fluid
channel. As particles flow through the channel (e.g., from blood), the nanowire sensors pick up the molecular signatures of the particles and relay this information through a connection of electrodes outside the body. Nanowires have potential to be used to detect the presence of altered genes
associated with cancer (NCI, 2006).

\subsection{Cantilevers }Nanoscaled cantilevers like springs arebeing developed using electron-beam lithography for an ultra sensitive bioassay. The flexible nature of the technology has the potential to offer high-throughput detection of proteins, DNA and RNA for a broad range of applications ranging from disease diagnosis to biological weapons detection. through e-beam lithography (Klein et al.2005). Nano-scale cantilevers resemble an everyday comb with evenly spaced teeth. The cantilevers possess conductive properties and are coated with specific antibodies responsive to cancer proteins (NCI). Protein secreted from cancerous cells attaches to the antibodies bonded to the cantilevers and actually cause the teeth to bend. This deformation creates a change in conductivity in the cantilever. This change can be measured in real time and the concentration of different molecular secretions determined (NCI, 2006), alerting doctors to the presence of cancer within a patient (NCI). This is much more effective than traditional detection methods because it allows doctors to detect cancer before tumor formation, and could in fact allow for the prevention of tumors if the disease is treated appropriately. 



\printbibliography
\end{document}
