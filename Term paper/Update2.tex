\documentclass{article}
\usepackage[utf8]{inputenc}
\usepackage{graphics}
\usepackage[paper=a4paper,left=3cm, right=3cm, top=2cm]{geometry}
\usepackage{blindtext}
\usepackage{multicol}
\usepackage{pgfplots}
\usepackage{float}
\usepackage{biblatex} %Imports biblatex package
\usepackage{amsmath}
\addbibresource{minor.bib} %Import the bibliography file


\pgfplotsset{compat = newest}


\title{%
Nuclear Medicine: An Approach Via Nanotechnology\\
   \large Department of Biomedical Engineering\\
    National Institute of Technology, Raipur}
\author{Durgesh Kumar(18111023)}
\date{January 2022}

\begin{document}

\maketitle

\begin{abstract}\textbf{The birth of nanotechnology in human society was around 2000 years ago and soon found applications in various fields and with nanotechnology nuclear medicine (radiation techniques) are an important tool in fighting against cancer and are used to treat for a variety of malignant tumors of different origins and stage; it is used in the treatment of as many as 50\% of all cancer patients and other then cancer infection/inflammation, and brain and heart diseases also treated with it.Every day in the clinic, nuclear physicians work with nuclear biologists, chemists, and physicists to practise nuclear medicine. Nanomedicine is a medical use of nanomaterials, biologic and nanoelectronic devices, biosensors, and possibly molecular nanotechnology. The fact that radiation dose can be supplied locally and that cells within the radiation nanoparticles transfer radionuclides to tumours while sparing healthy tissues from radiation induced damage account for radiation's efficacy as a cancer treatment technique.Because nanoparticles can be targeted to malignancies, nanotechnology has a promising future in radiotherapy. As a result, scientists have seized on the notion of employing nanoparticles to deliver radionuclides to tumours while avoiding radiation damage to healthy cells. Dendrimer nanocomposites have recently been discovered to be useful in radiation and imaging of tumour microvasculature. Carbon nanoparticles, also known as buckyballs, are barely a nanometer in diameter and could be used as antiradiation medications in the future to assist protect against the negative effects of cancer therapies or dirty bombs.Radiation therapy and chemotherapy frequently harm cells and tissues by releasing potentially harmful reactive oxygen species such as free radicals, oxygen ions, and peroxides. The electron clouds that surround buckyballs are thought to be capable of absorbing these free radicals. In light of a number of recent breakthroughs, the emerging role of nanotechnology in conjunction with radiation biology appears to be highly promising.}\\ 
    \\
    \textbf{\textit{Keywords-Cancer treatment; dendrimer nanocomposites;Nano Technology }}
\end{abstract}
