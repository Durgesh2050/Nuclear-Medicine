\documentclass{article}
\usepackage[utf8]{inputenc}
\usepackage{graphics}
\usepackage[paper=a4paper,left=3cm, right=3cm, top=2cm]{geometry}
\usepackage{blindtext}
\usepackage{multicol}
\usepackage{pgfplots}
\usepackage{float}
\usepackage{biblatex} %Imports biblatex package
\usepackage{amsmath}
\addbibresource{minor.bib} %Import the bibliography file


\pgfplotsset{compat = newest}


\title{%
Nuclear Medicine: An Approach Via Nanotechnology\\
   \large Department of Biomedical Engineering\\
    National Institute of Technology, Raipur}
\author{Durgesh Kumar(18111023)}
\date{January 2022}

\begin{document}

\maketitle

\begin{abstract}\textbf{The birth of nanotechnology in human society was around 2000 years ago and soon found applications in various fields and with nanotechnology nuclear medicine (radiation techniques) are an important tool in fighting against cancer and are used to treat for a variety of malignant tumors of different origins and stage; it is used in the treatment of as many as 50\% of all cancer patients. stems from the facts that radiation dose can be delivered locally and that cells within the radiation nanoparticles to deliver radionuclides to tumors and sparing healthy tissues from radiation induced damage.}\\ 
    \\
    \textbf{\textit{Keywords-Cardiovascular disease; Machine learning; Random forest }}
\end{abstract}


\end{document}
